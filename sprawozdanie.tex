\documentclass{article}
\usepackage{graphicx}
\usepackage{amsmath}
\usepackage{polski}
\usepackage[utf8]{inputenc}
\usepackage{hyperref}

\hypersetup{
    colorlinks=true,
    linkcolor=blue,
    filecolor=magenta,
    urlcolor=cyan,
}

\title{Symulowanie Procesów Losowych - Kule i Urny}
\author{Jakub Kogut}
\date{}

\begin{document}

\maketitle

\section{Wstęp}
Sprawozdanie do zadania domowego 2.
\section{Opis Zadania}
Należało zaprojektować oraz zaimplementować probalistyczny model kul i urn. Kolejne m kul wrzucanych jest niezależnie oraz jednostajnie do n ponumernowanych urn. Celem zadania było eksperymentalne wyznaczenie następujących wartości:
\begin{enumerate}
    \item $B_n$ - moment pierwszej kolizji
    \item $U_n$ - liczba urn, które pozostają puste po wrzuceniu $n$ kul
    \item $C_n$ - minimalna liczba rzutów, po której w każdej z urn znajduje się co najmniej jedna kula
    \item $D_n$ - minimalna liczba rzutów, po której w każdej z urn znajduje się co najmniej dwie kule (równocześnie jest to warunek końcowy symulacji)
    \item $D_n - C_n$ - ilość rzutów potrzebna do uzyskania drugiej kuli w każdej z urn, po tym jak w każdej jest już minimum jedna kula
\end{enumerate}
\section{Metodologia}
Podana była ustalona wartość $k=50$ niezależnych symulacji dla każdego $n$ z zakresu $n \in \{1000, 2000, ..., 100000\}$. W trakcie wykonywania jej, na bierząco zapisywane były szukane wartości. Następnie wypisywano wyniki do pliku, liczono średnie oraz generowano wykresy.\newline
Przy generowaniu liczb losowych użyłem generatora \textit{Mersenne Twister}
\section{Wnioski}
Na podstawie przeprowadzonych symulacji i wygenerowanych wykresów można wyciągnąć następujące wnioski dotyczące badanych zmiennych:

\subsection*{1. Moment pierwszej kolizji ($B_n$)}
Wartości momentu pierwszej kolizji rosną w miarę zwiększania liczby urn, co jest zgodne z intuicją wynikającą z paradoksu urodzinowego. \hyperref[fig:WykresBn]{Wykres średnich} wartości $B_n$ względem $n$ wskazuje na niską koncentrację wyników wokół wartości średnich, zmniejszającą się wraz ze wzrostem $n$. \newline
\hyperref[fig:WykresBnratio]{Wykres ciągu $B_n$ podzielonego przez podane funkcje $n$ oraz $\sqrt{n}$} sugeruje, że asymptotyka $B_n$ jest zbliżona do $O(n)$.

\subsection*{2. Liczba pustych urn ($U_n$)}
Liczba pustych urn po wrzuceniu $n$ kul zwiększa się wraz ze wzrostem $n$, dobrze odzwierciedlając wykres $y=n$. \hyperref[fig:WykresUn]{Wykres $U_n$} przedstawia wartości poszczególnych symulacji mocno skoncentrowane wokół średniej. \hyperref[fig:WykresUnratio]{Wykresy ilorazu $U_n/n$} wskazują, że wartości te są asymptotycznie zbliżone do stałej wartości $O(n)$.

\subsection*{3. Minimalna liczba rzutów do zapełnienia urn ($C_n$)}
Na \hyperref[fig:WykresCn]{wykresie $C_n$} widoczna jest dobra koncentraca wyników poszczególnych symulacji do wartości średniej, choć zmniejszającą się wraz z $n$. \hyperref[fig:WykresCnratio]{Wykresy ilorazów $\frac{C_n}{n}$, $\frac{C_n}{\ln n}$} wskazują, że wartości te są asymptotycznie zbliżone do stałej wartości $O(n)$.

\subsection*{4. Minimalna liczba rzutów do dwóch kul w każdej urnie ($D_n$)}
Aby każda urna zawierała przynajmniej dwie kule, potrzebna jest większa liczba rzutów niż w przypadku $C_n$. Wyniki symulacji pokazują, że wartości $D_n$ rosną szybciej niż $C_n$. \hyperref[fig:WykresDn]{Wykres $D_n$} pokazuje, że wartości $D_n$ są podobnie skoncentrowane wokół wartości średnich jak $C_n$. \hyperref[fig:WykresDnratio]{Z Wykresów ilorazów $\frac{D_n}{n}$, $\frac{D_n}{n \ln n}$, $\frac{D_n}{n \ln \ln n}$} można jednoznaczni stwierdzić, że wartości $D_n$ są asymptotycznie zbliżone do $O(n \ln n)$.

\subsection*{5. Różnica między $D_n$ a $C_n$}
Wartości $D_n - C_n$ reprezentują dodatkowy wysiłek wymagany, aby każda urna zawierała co najmniej dwie kule po osiągnięciu pełnego zapełnienia przynajmniej jedną kulą. \hyperref[fig:WykresDnCn]{Wykres $D_n - C_n$} pokazuje, że wartości te rosną w miarę zwiększania $n$. \hyperref[fig:WykresDn-Cn]{Wykresy ilorazów $\frac{D_n - C_n}{n}$, $\frac{D_n - C_n}{n \ln n}$, $\frac{D_n - C_n}{n \ln \ln n}$}  nie jedoznacznie sugerują, że wartości te są asymptotycznie zbliżone do $O(n \ln n)$ lub $O(n \ln \ln n)$.


\subsection*{6. Intuicje związane z nazwami problemów}
\begin{enumerate}
    \item \textbf{Paradox urodzinowy ($B_n$):} Intuicja opiera się na zaskakująco małej liczbie kul potrzebnych do pierwszej kolizji, co jest analogiczne do problemu dwóch osób dzielących tę samą datę urodzin.
    \item \textbf{Problem kolekcjonera kuponów ($C_n$):} Minimalna liczba rzutów do zapełnienia urn przypomina problem zebrania pełnego zestawu kuponów w przypadku losowego wybierania.
\end{enumerate}

\subsection*{7. Znaczenie w kontekście funkcji haszujących}
Paradox urodzinowy ma istotne zastosowanie w kryptografii, szczególnie w kontekście funkcji haszujących. Mała liczba rzutów prowadząca do kolizji podkreśla konieczność stosowania funkcji haszujących o dużej przestrzeni wyjściowej w celu minimalizacji ryzyka kolizji.

\section{Podsumowanie}
Przeprowadzone symulacje potwierdzają teoretyczne właściwości modelu kul i urn, w tym asymptotykę badanych zmiennych oraz intuicje związane z ich nazewnictwem. Dodatkowo, zastosowanie generatora Mersenne Twister zapewniło stabilność i wiarygodność wyników.

\pagebreak

\section{Wykresy}
\begin{figure}[ht]
    \centering
    \includegraphics[width=0.8\textwidth]{wykresy/B(n).png}
    \caption{Wykres Wykreswartości $B_n$}
    \label{fig:WykresBn}
\end{figure}
\begin{figure}[ht]
    \centering
    \includegraphics[width=0.8\textwidth]{wykresy/U(n).png}
    \caption{Wykres wartości $U_n$}
    \label{fig:WykresUn}
\end{figure}
\begin{figure}[ht]
    \centering
    \includegraphics[width=0.8\textwidth]{wykresy/C(n).png}
    \caption{Wykres wartości $C_n$}
    \label{fig:wykresCn}
\end{figure}
\begin{figure}[ht]
    \centering
    \includegraphics[width=0.8\textwidth]{wykresy/D(n).png}
    \caption{Wykres wartości $D_n$}
    \label{fig:WykresDn}
\end{figure}
\begin{figure}[ht]
    \centering
    \includegraphics[width=0.8\textwidth]{wykresy/Dn-Cn.png}
    \caption{Wykres wartości $D_n - C_n$}
    \label{fig:WykresDnCn}
\end{figure}
\begin{figure}[ht]
    \centering
    \includegraphics[width=0.8\textwidth]{wykresy/Bnratio.png}
    \caption{Wykres wartości $B_n$ podzielone przez $n$ oraz $sqrt(n)$}
    \label{fig:WykresBnratio}
\end{figure}
\begin{figure}[ht]
    \centering
    \includegraphics[width=0.8\textwidth]{wykresy/Unratio.png}
    \caption{Wykres wartości $U_n$ podzielone przez $n$}
    \label{fig:WykresUnratio}
\end{figure}
\begin{figure}[ht]
    \centering
    \includegraphics[width=0.8\textwidth]{wykresy/Cnratio.png}
    \caption{Wykres wartości $C_n$ podzielone przez $n$, $n \ln n$ oraz $n^2$}
    \label{fig:WykresCnratio}
\end{figure}
\begin{figure}[ht]
    \centering
    \includegraphics[width=0.8\textwidth]{wykresy/Dnratio.png}
    \caption{Wykres wartości $D_n$ podzielone przez $n$, $n \ln n$, $n \ln \ln n$}
    \label{fig:WykresDnratio}
\end{figure}
\begin{figure}[ht]
    \centering
    \includegraphics[width=0.8\textwidth]{wykresy/DnCnratio.png}
    \caption{Wykres wartości $D_n - C_n$ podzielone przez $n$, $n \ln n$, $n \ln \ln n$}
    \label{fig:WykresDn-Cn}
\end{figure}



\end{document}
