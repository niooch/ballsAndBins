\documentclass{article}
\usepackage{graphicx}
\usepackage{amsmath}
\usepackage{polski}
\usepackage[utf8]{inputenc}

\title{Symulowanie Procesów Losowych - Kule i Urny}
\author{Jakub Kogut}
\date{}

\begin{document}

\maketitle

\section{Wstęp}
Sprawozdanie do zadania domowego 2.
\section{Opis Zadania}
Należało zaprojektować oraz zaimplementować probalistyczny model kul i urn. Kolejne m kul wrzucanych jest niezależnie oraz jednostajnie do n ponumernowanych urn. Celem zadania było eksperymentalne wyznaczenie następujących wartości:
\begin{enumerate}
    \item $B_n$ - moment pierwszej kolizji
    \item $U_n$ - liczba urn, które pozostają puste po wrzuceniu $n$ kul
    \item $C_n$ - minimalna liczba rzutów, po której w każdej z urn znajduje się co najmniej jedna kula
    \item $D_n$ - minimalna liczba rzutów, po której w każdej z urn znajduje się co najmniej dwie kule (równocześnie jest to warunek końcowy symulacji)
    \item $D_n - C_n$ - ilość rzutów potrzebna do uzyskania drugiej kuli w każdej z urn, po tym jak w każdej jest już minimum jedna kula
\end{enumerate}
\section{Metodologia}
Podana była ustalona wartość $k=50$ niezależnych symulacji dla każdego $n$ z zakresu $n \in \{1000, 2000, ..., 100000\}$. W trakcie wykonywania jej, na bierząco zapisywane były szukane wartości. Następnie wypisywano wyniki do pliku, liczono średnie oraz generowano wykresy.\newline
Przy generowaniu liczb losowych użyłem generatora \textit{Mersenne Twister}
\section{Wnioski}
Na podstawie przeprowadzonych symulacji i wygenerowanych wykresów można wyciągnąć następujące wnioski dotyczące badanych zmiennych:

\subsection*{1. Moment pierwszej kolizji ($B_n$)}
Wartości momentu pierwszej kolizji rosną w miarę zwiększania liczby urn, co jest zgodne z intuicją wynikającą z paradoksu urodzinowego. Wykres średnich wartości $B_n$ względem $n$ sugeruje, że średnia asymptotycznie zmierza do funkcji proporcjonalnej do $\sqrt{n}$, co potwierdza teoretyczne przewidywania.

\subsection*{2. Liczba pustych urn ($U_n$)}
Liczba pustych urn po wrzuceniu $n$ kul zmniejsza się wraz ze wzrostem $n$, co odzwierciedla oczekiwaną koncentrację kul w większej liczbie urn. Wykresy ilorazu $U_n/n$ wskazują, że liczba pustych urn proporcjonalna jest do wartości $e^{-n/n}$, zgodnie z teorią.

\subsection*{3. Minimalna liczba rzutów do zapełnienia urn ($C_n$)}
Problem kolekcjonera kuponów sugeruje, że minimalna liczba rzutów, po której każda urna zawiera przynajmniej jedną kulę, jest asymptotycznie proporcjonalna do $n \ln n$. Na wykresach widoczna jest zgodność wyników eksperymentalnych z przewidywaniami teoretycznymi.

\subsection*{4. Minimalna liczba rzutów do dwóch kul w każdej urnie ($D_n$)}
Aby każda urna zawierała przynajmniej dwie kule, liczba rzutów rośnie znacząco szybciej. Analiza wskazuje, że asymptotyka $D_n$ jest zbliżona do $n \ln n$ z dodatkowymi wyrazami zależnymi od stałych wyższych rzędów.

\subsection*{5. Różnica między $D_n$ a $C_n$}
Wartości $D_n - C_n$ reprezentują dodatkowy wysiłek wymagany, aby każda urna zawierała co najmniej dwie kule po osiągnięciu pełnego zapełnienia. Wyniki symulacji pokazują, że różnica ta jest asymptotycznie proporcjonalna do $n$.

\subsection*{6. Koncentracja wyników}
Wartości badanych zmiennych w poszczególnych próbach koncentrują się wokół wartości średnich, co wskazuje na małą wariancję generatora liczb losowych Mersenne Twister. Wartości średnie dobrze przybliżają teoretyczne wyniki.

\subsection*{7. Intuicje związane z nazwami problemów}
\textbf{Paradox urodzinowy ($B_n$):} Intuicja opiera się na zaskakująco małej liczbie kul potrzebnych do pierwszej kolizji, co jest analogiczne do problemu dwóch osób dzielących tę samą datę urodzin.
\textbf{Problem kolekcjonera kuponów ($C_n$):} Minimalna liczba rzutów do zapełnienia urn przypomina problem zebrania pełnego zestawu kuponów w przypadku losowego wybierania.

\subsection*{8. Znaczenie w kontekście funkcji haszujących}
Paradox urodzinowy ma istotne zastosowanie w kryptografii, szczególnie w kontekście funkcji haszujących. Mała liczba rzutów prowadząca do kolizji podkreśla konieczność stosowania funkcji haszujących o dużej przestrzeni wyjściowej w celu minimalizacji ryzyka kolizji.

\section{Podsumowanie}
Przeprowadzone symulacje potwierdzają teoretyczne właściwości modelu kul i urn, w tym asymptotykę badanych zmiennych oraz intuicje związane z ich nazewnictwem. Dodatkowo, zastosowanie generatora Mersenne Twister zapewniło stabilność i wiarygodność wyników.

\section{Wykresy}
\begin{figure}[h]
    \centering
    \includegraphics[width=0.8\textwidth]{wykresy/B(n).png}
    \caption{Wykres wartości $B_n$}
\end{figure}
\begin{figure}[h]
    \centering
    \includegraphics[width=0.8\textwidth]{wykresy/U(n).png}
    \caption{Wykres wartości $U_n$}
\end{figure}
\begin{figure}[h]
    \centering
    \includegraphics[width=0.8\textwidth]{wykresy/C(n).png}
    \caption{Wykres wartości $C_n$}
\end{figure}
\begin{figure}[h]
    \centering
    \includegraphics[width=0.8\textwidth]{wykresy/D(n).png}
    \caption{Wykres wartości $D_n$}
\end{figure}
\begin{figure}[h]
    \centering
    \includegraphics[width=0.8\textwidth]{wykresy/D(n) - C(n).png}
    \caption{Wykres wartości $D_n - C_n$}
\end{figure}



\end{document}
